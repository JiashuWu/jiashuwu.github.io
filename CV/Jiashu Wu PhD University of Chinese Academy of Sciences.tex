%*****************************%
%      Author: Jiashu Wu      %
% Website: jiashuwu.github.io %
%*****************************%

\documentclass[letterpaper,10.9pt]{article}

\usepackage{latexsym}
\usepackage[empty]{fullpage}
\usepackage{titlesec}
\usepackage[usenames,dvipsnames]{color}
\usepackage{hyperref}
\usepackage{fancyhdr}
\usepackage{marvosym}
\usepackage{geometry}
\usepackage{setspace}
\usepackage{multicol}
\usepackage{graphicx}
\usepackage{multirow}
\geometry{margin=39pt}
\singlespacing

\raggedright
\setlength{\tabcolsep}{0in}

\titleformat{\section}{
  \vspace{-4pt}\scshape\raggedright\large
}{}{0em}{}[\color{black}\titlerule \vspace{-5pt}]

\begin{document}

% Non photo version
\iftrue
\begin{tabular*}{\textwidth}{l@{\extracolsep{\fill}}r}
  \huge \textbf{Jiashu Wu} & Phone/WeChat: \href{tel:17801323125}{17801323125}\vspace{2pt}\\
  \Large Ph.D. University of Chinese Academy of Sciences & Email: \href{mailto:wujiashu21@mails.ucas.ac.cn}{wujiashu21@mails.ucas.ac.cn}\vspace{2pt}\\
  DoB: 13\textsuperscript{th} Jun 1997, Hometown: Beijing, China & Homepage: \href{https://jiashuwu.github.io}{jiashuwu.github.io}\\
\end{tabular*}
\fi

% Photo version
\iffalse
\begin{tabular*}{\textwidth}{l@{\extracolsep{\fill}}r}
   & \multirow{4}{*}{\includegraphics[width=2cm,keepaspectratio]{WUJIASHU.jpg}}\\
  \huge \textbf{Jiashu Wu}\vspace{2pt}\\
  \Large Ph.D. University of Chinese Academy of Sciences\vspace{2pt}\\
  DoB: 13\textsuperscript{th} Jun 1997, Hometown: Beijing, China\\
  Phone/WeChat: \href{tel:17801323125}{17801323125}, Email: \href{mailto:wujiashu21@mails.ucas.ac.cn}{wujiashu21@mails.ucas.ac.cn}, Homepage: \href{https://jiashuwu.github.io}{jiashuwu.github.io}\\
\end{tabular*}
\fi



%===================

\section{\textbf{Education}}
\textbf{University of Chinese Academy of Sciences} \hfill Beijing \& Shenzhen, China\\
\textbf{Doctor of Philosophy in Computer Science} \hfill Sept 2021 - Jul 2024\\
Research topic: Domain adaptation-based IoT intrusion detection, Advisor: Prof. Yang Wang, GPA: 3.98 (91.7)\\
Faculty: Shenzhen Institute of Advanced Technology, Chinese Academy of Sciences

\vspace{8pt}

\textbf{University of Melbourne} \hfill Melbourne, Australia\\
\textbf{Master of Information Technology (with Distinction)} \hfill Mar 2019 - Dec 2020\\
Major: Artificial Intelligence, Advisor: Prof. Rui Zhang, GPA: 4.0 (88.1, First Class Honour, top 2\%)

\vspace{8pt}

\textbf{University of Sydney} \hfill Sydney, Australia\\
\textbf{Bachelor of Science} \hfill Feb 2016 - Dec 2018\\
Double Major: Computer Science, Financial Mathematics \& Statistics, Advisor: Prof. Simon Poon\\
GPA: 3.96 (86.5, High Distinction, top 2\%)

\vspace{8pt}

\textbf{Beijing Institute of Technology} \hfill Beijing, China\\
Major: Software Engineering, transferred to USYD in 2016 (Verifiable via Chsi.com) \hfill Sept 2015 - Jan 2016\\



%===================

\section{\textbf{Research Interests}}

Tackling IoT intrusion detection via domain adaptation. Design accurate and efficient algorithms for data-scarce scenarios such as unsupervised and open-set DA. Challenges including domain heterogeneities and the avoidance of under-transfer and negative-transfer. 

\textbf{Key skills}: Network data analytics, intrusion detection (IDS), feature engineering, transfer learning, statistical analysis, performance optimisation, academic paper and patent writing. 



%===================

\section{\textbf{Publications}}

I have published \textbf{9 CCF-A/JCR Q1 papers} and have 3 IEEE/ACM Trans/CCF-A papers under review. I have \textbf{8 patents being granted} and 11 patents under examination. \textbf{Key topics}: Network data analytics, transfer learning, IoT security, intrusion detection, data caching and storage, data stream analysis, etc. Below are selected publications. 

\begin{enumerate}
  \item Adaptive Bi-recommendation and Self-improving Network for Heterogeneous Domain Adaptation-assisted IoT Intrusion Detection\\
  \textbf{Jiashu Wu}, Yang Wang\textsuperscript{\Letter}, Hao Dai, Chengzhong Xu, Kenneth B. Kent\\
  \textit{IEEE Internet of Things Journal} (\textbf{IEEE IoTJ}, \textbf{JCR Q1}, \textbf{IF=10.6}), 2023

  \item Heterogeneous Domain Adaptation for IoT Intrusion Detection: A Geometric Graph Alignment Approach\\
  \textbf{Jiashu Wu}, Hao Dai, Yang Wang\textsuperscript{\Letter}, Kejiang Ye, Chengzhong Xu\\
  \textit{IEEE Internet of Things Journal} (\textbf{IEEE IoTJ}, \textbf{JCR Q1}, \textbf{IF=10.6}), 2023

  \item Cost-Efficient Sharing Algorithms for DNN Model Serving in Mobile Edge Networks\\
  Hao Dai, \textbf{Jiashu Wu}, Yang Wang\textsuperscript{\Letter}, Jerome Yen, Yong Zhang, Chengzhong Xu\\
  \textit{IEEE Transactions on Services Computing} (\textbf{IEEE TSC}, \textbf{CCF-A}, \textbf{IF=11.0}), 2023

  \item Joint Semantic Transfer Network for IoT Intrusion Detection\\
  \textbf{Jiashu Wu}, Yang Wang\textsuperscript{\Letter}, Binhui Xie, Shuang Li, Hao Dai, Kejiang Ye, Chengzhong Xu\\
  \textit{IEEE Internet of Things Journal} (\textbf{IEEE IoTJ}, \textbf{JCR Q1}, \textbf{IF=10.6}), 2022

  \item PackCache: An Online Cost-driven Data Caching Algorithm in the Cloud\\
  \textbf{Jiashu Wu}, Hao Dai, Yang Wang\textsuperscript{\Letter}, Yong Zhang, Dong Huang, Chengzhong Xu\\
  \textit{IEEE Transactions on Computers} (\textbf{IEEE TC}, \textbf{CCF-A}, \textbf{IF=3.7}), 2022

  \item Simultaneous Semantic Alignment Network for Heterogeneous Domain Adaptation\\
  Shuang Li, Binhui Xie, \textbf{Jiashu Wu}, Ying Zhao, Chi Harold Liu\textsuperscript{\Letter}, Zhengming Ding\\
  \textit{ACM International Conference on Multimedia} (\textbf{ACM MM}, \textbf{CCF-A}), 2020, Seattle, WA, USA

  \item Towards Scalable and Efficient Deep-RL in Edge Computing : A Game-based Partition Approach\\
  Hao Dai, \textbf{Jiashu Wu}, Yang Wang\textsuperscript{\Letter}, Chengzhong Xu\\
  \textit{Journal of Parallel and Distributed Computing} (\textbf{JPDC}, \textbf{JCR Q1}, \textbf{IF=3.8}), 2022

  \item Open Set Dandelion Network for IoT Intrusion Detection\\
  \textbf{Jiashu Wu}, Hao Dai, Yang Wang\textsuperscript{\Letter}, Kenneth B. Kent, Chengzhong Xu\\
  Under review at \textit{ACM Transactions on Internet Technology} (\textbf{ACM TOIT}, \textbf{JCR Q1}, \textbf{IF=5.3}), 2023

  \item HI-CPT: Towards Verifiable IoT Intrusion Detection under Data-scarce Heterogeneous Environment\\
  \textbf{Jiashu Wu}, Hao Dai, Yang Wang\textsuperscript{\Letter}, Kejiang Ye, Chengzhong Xu\\
  Under review at \textit{IEEE Transactions on Cybernetics} (\textbf{IEEE TCYB}, \textbf{JCR Q1}, \textbf{IF=11.8}), 2023
\end{enumerate}



%===================

\section{\textbf{Project and Internship Experience}}

\textbf{Project and Internship topics}: Network data analytics, transfer learning, intrusion detection, model caching and distributed training, data stream analysis, secure and efficient data storage, etc. 

\textbf{Applications}: Information security, big data storage \& analytics, IoT, data analytics, etc. Below are selected projects. 

\vspace{2mm}

\textbf{Research on IoT Intrusion Detection via Domain Adaptation Approach}

Ph.D. research topic; Internship at BIT during Nov 2019 - Feb 2020

\begin{itemize}
  \setlength\itemsep{1.8pt}
  \item Tackle IoT intrusion detection via domain adaptation. Propose 5 algorithms targeting scenarios with diverse data scarcity. Tackle domain heterogeneity and negative transfer via self supervision, probabilistic semantics, etc. 
  \item The proposed algorithms improve IoT intrusion detection accuracy by 4\%-17\%, the efficacy of proposed mechanisms are statistically verified. Low latency makes the proposed algorithms feasible for IoT security monitoring. 
  \item \textbf{Key skills}: network data analytics, transfer learning, feature engineering, Python programming, performance optimisation and academic writing. Published 5 CCF-A/JCR Q1 papers and 4 patents. 
\end{itemize}

\vspace{1pt}

\textbf{Online Cost-driven Data Caching Algorithm in the Cloud} \hfill National Research\&Development Project at CAS

\begin{itemize}
  \setlength\itemsep{1.8pt}
  \item Design and implement an online cost-driven data caching algorithm in the distributed cloud environment. Challenges including cost optimisation under online setting. Solved via packable anticipatory caching model construction. The algorithm is feasible for big data applications due to its optimised cost and excellent scalability. 
  \item The algorithm reduces data caching cost by 5\%-11\%. Theoretically, the competitive ratio and its lower bound for the online algorithm is proved. 
  \item \textbf{Key skills}: caching optimisation, theoretical analysis, data mining and analytics, Python programming. Published 2 CCF-A papers and 3 patents. 
\end{itemize}

\vspace{1pt}

\textbf{Multi-indexing System based on HDFS for Remote Sensing Data Storage}

National Research\&Development Project at CAS; Internship at CAS during Nov 2020 - Aug 2021

\begin{itemize}
  \setlength\itemsep{1.8pt}
  \item Design a multi-indexing remote sensing data storage and analytics system based on HDFS. With low latency, the system benefits geospatial data storage and analytics, and is broadly applicable under various scenarios. 
  \item The multi-indexing mechanism reduces the indexing and querying time by 60\%. Besides, the system is immune to data loss and is scalable and resource-efficient.
  \item \textbf{Key skills}: big data storage system design and indexing algorithm design. Published 1 JCR Q1 paper. 
\end{itemize}



%===================

\section{\textbf{Award}}
\vspace{-5mm}
\begin{multicols}{2}
\begin{itemize}
  \setlength\itemsep{1.8pt}
  \item President Scholarship of CAS, 2023\\(\textbf{Highest award for PhD in CAS, top 0.5\%})
  \item Pacemaker for Outstanding Student, UCAS, 2023
  \item Outstanding Student, UCAS, 2023
  \item Dean's Honours List, University of Melbourne, 2019
  \item Dean's List of Excellence in Academic Performance, University of Sydney, 2017 \& 2018
\end{itemize}
\end{multicols}



%===================

\section{\textbf{Skills and Language Ability}}
Programming Skill: Python (PyTorch, Sklearn, etc), Java, SQL. 

Technical Skill: Computer network, data analytics, statistics, database, operating system, etc. 

Language Skill: Passed IELTS Academic (score=7.0), CET-4 (score=665, got full mark in reading), lived and studied in Australia for 5 years, comfortable in English communication environment. Native speaker of Mandarin Chinese. 

Writing Skill: Write logically and professionally, have strong capability of academic paper, patent and technical report writing. 

\end{document}
