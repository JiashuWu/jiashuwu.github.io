%*****************************%
%      Author: Jiashu Wu      %
% Website: jiashuwu.github.io %
%*****************************%

\documentclass[letterpaper,11pt]{article}

\usepackage{latexsym}
\usepackage[empty]{fullpage}
\usepackage{titlesec}
\usepackage[usenames,dvipsnames]{color}
\usepackage{hyperref}
\usepackage{fancyhdr}
\usepackage{marvosym}
\usepackage{geometry}
\geometry{margin=39pt}

\raggedright
\setlength{\tabcolsep}{0in}

\titleformat{\section}{
  \vspace{-4pt}\scshape\raggedright\large
}{}{0em}{}[\color{black}\titlerule \vspace{-5pt}]

\newcommand{\RNum}[1]{\uppercase\expandafter{\romannumeral #1\relax}}

\begin{document}

\begin{tabular*}{\textwidth}{l@{\extracolsep{\fill}}r}
  \textbf{\Large Jiashu Wu} & Homepage: \href{https://jiashuwu.github.io}{jiashuwu.github.io} \\
  Ph.D. Student, University of Chinese Academy of Sciences & Email: \href{mailto:wujiashu21@mails.ucas.ac.cn}{wujiashu21@mails.ucas.ac.cn}\\
  DoB: 13\textsuperscript{th} Jun 1997 & Mobile: \href{tel:8617801323125}{+86-17801323125} \\
\end{tabular*}

\vspace{1pt}

\section{Education}
\textbf{University of Chinese Academy of Sciences} \hfill Beijing \& Shenzhen, China\\
\textbf{Doctor of Philosophy in Computer Science} \hfill Sept 2021 - Jun 2024 (Expected)\\
Research topic: Domain adaptation-based IoT intrusion detection, advisor: Prof. Yang Wang

\vspace{9pt}

\textbf{University of Melbourne} \hfill Melbourne, Australia\\
\textbf{Master of Information Technology (with Distinction)} \hfill Jan 2019 - Dec 2020\\
Major: Artificial Intelligence, advisor: Prof. Rui Zhang, average mark: 88.1 (First Class Honour, top 2\%)

\vspace{9pt}

\textbf{University of Sydney} \hfill Sydney, Australia\\
\textbf{Bachelor of Science} \hfill Jan 2016 - Dec 2018\\
Major: Computer Science, Financial Mathematics \& Statistics, advisor: Prof. Simon Poon, average mark: 86.5 (High Distinction, top 2\%)

\vspace{9pt}

\textbf{Beijing Institute of Technology} \hfill Beijing, China\\
Major: Software Engineering, transferred to USYD in 2016 \hfill Aug 2015 - Jan 2016\\


\vspace{1pt}

\section{Research Interests}

My research interest focuses on applying domain adaptation (DA) to enhance the efficacy of IoT intrusion detection. The DA techniques I studied including heterogeneous DA, multi-source DA, semi-supervised DA, unsupervised DA, open-set DA, etc. 

\vspace{1pt}

\section{Publications (Full list: \href{https://scholar.google.com/citations?user=wGgUbQkAAAAJ}{Google Scholar})}

I have published 3 CCF-A papers, 4 CCF-B/Tsinghua-B papers, and have 2 papers under review at IEEE/ACM transactions. I have 7 patents being granted and 12 patents under examination. 

\begin{enumerate}
  \item Adaptive Bi-Recommendation and Self-improving Network for Heterogeneous Domain Adaptation-assisted IoT Intrusion Detection\\
  \textbf{Jiashu Wu}, Yang Wang\textsuperscript{\Letter}, Hao Dai, Chengzhong Xu, Kenneth B. Kent\\
  \textit{IEEE Internet of Things Journal} (\textbf{IEEE IoTJ}, Tsinghua-B, JCR Q1, IF: 10.6), 2023

  \item Heterogeneous Domain Adaptation for IoT Intrusion Detection: A Geometric Graph Alignment Approach\\
  \textbf{Jiashu Wu}, Hao Dai, Yang Wang\textsuperscript{\Letter}, Kejiang Ye, Chengzhong Xu\\
  \textit{IEEE Internet of Things Journal} (\textbf{IEEE IoTJ}, Tsinghua-B, JCR Q1, IF: 10.6), 2023

  \item Cost-Efficient Sharing Algorithms for DNN Model Serving in Mobile Edge Networks\\
  Hao Dai, \textbf{Jiashu Wu}, Yang Wang\textsuperscript{\Letter}, Jerome Yen, Yong Zhang, Chengzhong Xu\\
  \textit{IEEE Transactions on Services Computing} (\textbf{IEEE TSC}, CCF-A, JCR Q1, IF: 11.0), 2023

  \item Joint Semantic Transfer Network for IoT Intrusion Detection\\
  \textbf{Jiashu Wu}, Yang Wang\textsuperscript{\Letter}, Binhui Xie, Shuang Li, Hao Dai, Kejiang Ye, Chengzhong Xu\\
  \textit{IEEE Internet of Things Journal} (\textbf{IEEE IoTJ}, Tsinghua-B, JCR Q1, IF: 10.6), 2022

  \item PackCache: An Online Cost-driven Data Caching Algorithm in the Cloud\\
  \textbf{Jiashu Wu}, Hao Dai, Yang Wang\textsuperscript{\Letter}, Yong Zhang, Dong Huang, Chengzhong Xu\\
  \textit{IEEE Transactions on Computers} (\textbf{IEEE TC}, CCF-A, JCR Q2, IF: 3.7), 2022

  \item Towards Scalable and Efficient Deep-RL in Edge Computing : A Game-based Partition Approach\\
  Hao Dai, \textbf{Jiashu Wu}, Yang Wang\textsuperscript{\Letter}, Chengzhong Xu\\
  \textit{Journal of Parallel and Distributed Computing} (\textbf{JPDC}, CCF-B, JCR Q1, IF: 4.5), 2022

  \item Simultaneous Semantic Alignment Network for Heterogeneous Domain Adaptation\\
  Shuang Li, Binhui Xie, \textbf{Jiashu Wu}, Ying Zhao, Chi Harold Liu\textsuperscript{\Letter}, Zhengming Ding\\
  \textit{ACM International Conference on Multimedia} (\textbf{ACM MM}, CCF-A, IS: 12.9), 2020, Seattle, Washing, USA

  \item Open Set Dandelion Network for IoT Intrusion Detection\\
  \textbf{Jiashu Wu}, Hao Dai, Yang Wang\textsuperscript{\Letter}, Kenneth B. Kent, Chengzhong Xu\\
  Under review at \textit{ACM Transactions on Internet Technology} (\textbf{ACM TOIT}, CCF-B, JCR Q1, IF: 5.3), 2023

  \item HI-CPT: A Heterogeneous IoT Intrusion Detection Testing Platform\\
  \textbf{Jiashu Wu}, Hao Dai, Yang Wang\textsuperscript{\Letter}, Kejiang Ye, Chengzhong Xu\\
  Under review at \textit{IEEE Transactions on Cybernetics} (\textbf{IEEE TCYB},CCF-B, JCR Q1, IF: 11.8), 2023
\end{enumerate}

\vspace{1pt}

\section{Award}
\begin{itemize}
  \item Outstanding Student, University of Chinese Academy of Sciences, 2022
  %\item Graduate with Distinction, University of Melbourne, 2020
  \item Dean's Honours List, University of Melbourne, 2019
  \item Dean's List of Excellence in Academic Performance, University of Sydney, 2018 \& 2017
\end{itemize}

\vspace{1pt}

\section{Experience}

\textit{Software Engineer at Melbourne eResearch Group, Melbourne Australia} \hfill Mar 2020 - July 2020

Develop a meeting diarisation Android app for research purposes. Key skills including Java programming, Git code management and the interaction with Google ML Speech API. 

\vspace{9pt}

\textit{Student Intern at Beijing Institute of Technology, Beijing China} \hfill Nov 2019 - May 2020

Carry out research on domain adaptation and propose a novel simultaneous semantic alignment network. Key skills including DL algorithm design, Python programming and research paper writing (work published in ACM MM'20). 

\vspace{9pt}

\textit{Research and Development Engineer at University of Melbourne, Melbourne Australia} \hfill Jul 2019 - Nov 2019

Develop an academia searching platform under the challenging data-scarce condition. Key skills including Python programming, NLP data processing and recommender system algorithm design. 

\vspace{9pt}

\textit{Dalyell Scholar Program at University of Sydney, Sydney Australia} \hfill Mar 2018 - Jul 2018

Carry out research on Parkinson disease detection based on drawing patterns and develop an Android app for drawing data collection. Key skills including Java programming and database management. 

\vspace{1pt}

\section{Skills}
Programming Skill: Python, Java, SQL

Language Skill: English (IELTS Academic 7.0, live and study in Australia for 5 years), Mandarin Chinese (Native Speaker)

\end{document}
