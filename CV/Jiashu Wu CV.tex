%*****************************%
%      Author: Jiashu Wu      %
% Website: jiashuwu.github.io %
%*****************************%

\documentclass[letterpaper,11pt]{article}

\usepackage{latexsym}
\usepackage[empty]{fullpage}
\usepackage{titlesec}
\usepackage[usenames,dvipsnames]{color}
\usepackage[pdftex]{hyperref}
\usepackage{fancyhdr}
\usepackage[misc]{ifsym}
\usepackage{geometry}
\geometry{margin=39pt}

\raggedright
\setlength{\tabcolsep}{0in}

\titleformat{\section}{
  \vspace{-4pt}\scshape\raggedright\large
}{}{0em}{}[\color{black}\titlerule \vspace{-5pt}]

\newcommand{\RNum}[1]{\uppercase\expandafter{\romannumeral #1\relax}}

\begin{document}

\begin{tabular*}{\textwidth}{l@{\extracolsep{\fill}}r}
  \textbf{\Large Jiashu Wu} & Homepage: \href{https://jiashuwu.github.io}{jiashuwu.github.io} \\
  Ph.D. Student, University of Chinese Academy of Sciences & Email: \href{mailto:wujiashu21@mails.ucas.ac.cn}{wujiashu21@mails.ucas.ac.cn}\\
  DoB: 13\textsuperscript{th} Jun 1997 & Mobile: \href{tel:8617801323125}{+86-178-0132-3125} \\
\end{tabular*}

\vspace{1pt}

\section{Education}
\textbf{University of Chinese Academy of Sciences} \hfill Beijing, China\\
Doctor of Philosophy in Computer Science, advisor: Prof. Yang Wang \hfill Sept 2021 - Jun 2024 (Expected)

\vspace{9pt}

\textbf{University of Melbourne} \hfill Melbourne, Australia\\
Master of Information Technology (with Distinction) \hfill Jan 2019 - Dec 2020\\
Major: artificial intelligence, advisor: Prof. Rui Zhang, average mark: 88.1 (First Class Honour, top 2\%)

\vspace{9pt}

\textbf{University of Sydney} \hfill Sydney, Australia\\
Bachelor of Science \hfill Jan 2016 - Dec 2018\\
Major: Computer Science, Financial Mathematics \& Statistics, advisor: Prof. Simon Poon, average mark: 86.5 (High Distinction, top 2\%)

\vspace{1pt}

\section{Research Interests}

My research interest focuses on applying domain adaptation to enhance the efficacy of IoT intrusion detection. 

\vspace{1pt}

\section{Publications ($^*$co-first author, full list: \href{https://scholar.google.com/citations?user=wGgUbQkAAAAJ}{Google Scholar})}

\RNum{1}: IoT Intrusion Detection via Domain Adaptation and its Efficiency Improvement
\begin{enumerate}
  \item Heterogeneous Domain Adaptation for IoT Intrusion Detection: A Geometric Graph Alignment Approach\\
  \textbf{Jiashu Wu}, Hao Dai, Yang Wang \Letter, Kejiang Ye, Chengzhong Xu\\
  \textit{IEEE Internet of Things Journal} (\textbf{IEEE IoTJ}, IF: 10.24), 2023

  \item Cost-Efficient Sharing Algorithms for DNN Model Serving in Mobile Edge Networks\\
  Hao Dai, \textbf{Jiashu Wu}, Yang Wang \Letter, Jerome Yen, Yong Zhang, Chengzhong Xu\\
  \textit{IEEE Transactions on Services Computing} (\textbf{IEEE TSC}, IF: 11.02), 2023

  \item Joint Semantic Transfer Network for IoT Intrusion Detection\\
  \textbf{Jiashu Wu}, Yang Wang \Letter, Binhui Xie, Shuang Li, Hao Dai, Kejiang Ye, Chengzhong Xu\\
  \textit{IEEE Internet of Things Journal} (\textbf{IEEE IoTJ}, IF: 10.24), 2022

  \item PackCache: An Online Cost-driven Data Caching Algorithm in the Cloud\\
  \textbf{Jiashu Wu}, Hao Dai, Yang Wang \Letter, Yong Zhang, Dong Huang, Chengzhong Xu\\
  \textit{IEEE Transactions on Computers} (\textbf{IEEE TC}, IF: 3.18), 2022

  \item Towards Scalable and Efficient Deep-RL in Edge Computing : A Game-based Partition Approach\\
  Hao Dai, \textbf{Jiashu Wu}, Yang Wang \Letter, Chengzhong Xu\\
  \textit{Journal of Parallel and Distributed Computing} (\textbf{JPDC}, IF: 4.54), 2022

  \item Multi-Scenario Bimetric-Balanced IoT Resource Allocation: An Evolutionary Approach\\
  \textbf{Jiashu Wu}, Hao Dai, Yang Wang \Letter, Zhiying Tu\\
  \textit{IEEE International Conference on High Performance Computing and Communications} (\textbf{IEEE HPCC}), 2022

  \item Simultaneous Semantic Alignment Network for Heterogeneous Domain Adaptation\\
  Shuang Li, Binhui Xie, \textbf{Jiashu Wu}, Ying Zhao, Chi Harold Liu \Letter, Zhengming Ding\\
  \textit{ACM Multimedia} (\textbf{ACM MM}, IS: 12.9), 2020, Seattle, Washing, USA

  \item Adaptive Bi-Recommendation and Self-improving Network for Heterogeneous Domain Adaptation-assisted IoT Intrusion Detection\\
  \textbf{Jiashu Wu}, Yang Wang \Letter, Hao Dai, Chengzhong Xu, Kenneth B. Kent\\
  Under review at \textit{IEEE Internet of Things Journal} (\textbf{IEEE IoTJ}, IF: 10.24), 2023

  \item Open Set Dandelion Network for IoT Intrusion Detection\\
  \textbf{Jiashu Wu}, Hao Dai, Yang Wang \Letter, Kenneth B. Kent, Chengzhong Xu\\
  Under review at \textit{ACM Transactions on Internet Technology} (\textbf{ACM TIOT}, IF: 3.99), 2023
\end{enumerate}

\iffalse
\subsubsection*{II: Other Topics: Edge-cloud Task and Resource Allocation}
\begin{enumerate}
  \setcounter{enumi}{9}
  \item Multi-Scenario Bimetric-Balanced IoT Resource Allocation: An Evolutionary Approach\\
  \textbf{Jiashu Wu}, Hao Dai, Yang Wang \Letter, Zhiying Tu\\
  \textit{IEEE International Conference on High Performance Computing and Communications} (\textbf{IEEE HPCC}), 2022

  \item PECCO: A Profit and Cost-oriented Computation Offloading Scheme in Edge-Cloud Environment with Improved Moth-flame Optimisation\\
  \textbf{Jiashu Wu}, Hao Dai, Yang Wang \Letter, Shigen Shen, Chengzhong Xu\\
  \textit{Concurrency and Computation: Practice and Experience} (\textbf{CCPE}), 2022
\end{enumerate}

\subsubsection*{III: Other Topics: Efficient Data Storage}
\begin{enumerate}
  \setcounter{enumi}{11}
  \item A Self-contained and Self-explanatory DNA Storage System\\
  Min Li$^*$, \textbf{Jiashu Wu$^*$}, Junbiao Dai, Qingshan Jiang, Qiang Qu, Xiaoluo Huang, Yang Wang \Letter\\
  \textit{Nature Scientific Reports} (\textbf{Nature SR}), 2021

  \item MIX-RS: A Multi-indexing System based on HDFS for Remote Sensing Data Storage\\
  \textbf{Jiashu Wu}, Jingpan Xiong, Hao Dai, Yang Wang \Letter, Chengzhong Xu\\
  \textit{Tsinghua Science and Technology} (\textbf{TST}), 2021

  \item How does SSD cluster perform for distributed file systems: An empirical study\\
  \textbf{Jiashu Wu}, Yang Wang \Letter, Jinpeng Wang, Hekang Wang, Taorui Lin\\
  Under review at \textit{Concurrency and Computation: Practice and Experience} (\textbf{CCPE}), 2023
\end{enumerate}

\subsubsection*{IV: Other Topics: Data Processing}
\begin{enumerate}
  \setcounter{enumi}{14}
  \item Towards Fast Theta-join: A Prefiltering and Amalgamated Partitioning Approach\\
  \textbf{Jiashu Wu}, Yang Wang \Letter, Xiaopeng Fan, Kejiang Ye, Chengzhong Xu\\
  \textit{Concurrency and Computation: Practice and Experience} (\textbf{CCPE}), 2021
\end{enumerate}
\fi

\RNum{2}: Thesis
\begin{enumerate}
  \setcounter{enumi}{9}
  \item Research on IoT Intrusion Detection via Domain Adaptation Approach\\
  \textit{Ph.D Thesis at the University of Chinese Academy of Sciences}

  \item Learning to Rank with Small Set of Ground Truth Data\\
  \textit{Master Thesis at the University of Melbourne}
\end{enumerate}

\RNum{3}: Patents

\vspace{8pt}

\hspace{6pt} Five Chinese patents already be granted and $14$ Chinese patents under examination

\vspace{1pt}

\section{Award}
\begin{itemize}
  \item Outstanding Student, University of Chinese Academy of Sciences, 2022
  \item Graduate with Distinction, University of Melbourne, 2020
  \item Dean's Honours List, University of Melbourne, 2019
  \item Dean's List of Excellence in Academic Performance, University of Sydney, 2018 \& 2017
\end{itemize}

\vspace{1pt}

\section{Experience}

\textit{Software Engineer at Melbourne eResearch Group} \hfill Mar 2020 - July 2020, Melbourne Australia

Develop a meeting diarisation Android app for research purposes. Key skills including Java programming, Git code management and the interaction with Google ML Speech API. 

\vspace{9pt}

\textit{Student Intern at Beijing Institute of Technology} \hfill Nov 2019 - May 2020, Beijing China

Carry out research on domain adaptation and propose a novel simultaneous semantic alignment network. Key skills including DL algorithm design, Python programming and research paper writing (work published in ACMMM'20). 

\vspace{9pt}

\textit{Research and Development Engineer at University of Melbourne} \hfill Jul 2019 - Nov 2019, Melbourne Australia

Develop an academia searching platform under the challenging data-scarce condition. Key skills including Python programming, NLP data processing and recommender system algorithm design. 

\vspace{9pt}

\textit{Dalyell Scholar Program at University of Sydney} \hfill Mar 2018 - Jul 2018, Sydney Australia

Carry out research on Parkinson disease detection based on drawing patterns and develop an Android app for drawing data collection. Key skills including Java programming and database management. 

\vspace{1pt}

\section{Skills}
Programming Skill: Python, Java, Matlab, SQL, R

Language Skill: English (IELTS Academic 7.0, live and study in Australia for 5 years), Mandarin Chinese (Native Speaker)

\end{document}
