%*****************************%
%      Author: Jiashu Wu      %
% Website: jiashuwu.github.io %
%*****************************%

\documentclass[letterpaper,11pt]{article}

\usepackage[UTF8]{ctex}
\usepackage{latexsym}
\usepackage[empty]{fullpage}
\usepackage{titlesec}
\usepackage[usenames,dvipsnames]{color}
\usepackage{hyperref}
\usepackage{fancyhdr}
\usepackage{marvosym}
\usepackage{geometry}
\usepackage{setspace}
\geometry{margin=39pt}
\singlespacing
\setCJKmainfont{STSongti-SC-Regular}

\raggedright
\setlength{\tabcolsep}{0in}

\titleformat{\section}{
  \vspace{-4pt}\scshape\raggedright\large
}{}{0em}{}[\color{black}\titlerule \vspace{-5pt}]

\newcommand{\RNum}[1]{\uppercase\expandafter{\romannumeral #1\relax}}

\begin{document}

%===================

\begin{tabular*}{\textwidth}{l@{\extracolsep{\fill}}r}
  \textbf{\huge 吴嘉澍} & 个人主页:\href{https://jiashuwu.github.io}{jiashuwu.github.io}\vspace{2pt}\\
  中国科学院大学工学博士 & 电子邮箱:\href{mailto:wujiashu21@mails.ucas.ac.cn}{wujiashu21@mails.ucas.ac.cn}\vspace{2pt}\\
  生日:1997年6月13日\hspace{5mm}籍贯:北京海淀 & 手机/微信:\href{tel:8617801323125}{+86-17801323125}\\
\end{tabular*}

\vspace{1pt}



%===================

\section{教育背景}
\textbf{中国科学院大学} \hfill 中国北京/深圳\\
\textbf{工学博士} \hfill 2021年9月 - 2024年6月\\
专业:计算机应用技术,论文题目:面向物联网入侵检测的领域自适应方法研究\\
研究院:中国科学院深圳先进院,导师:王洋教授

\vspace{9pt}

\textbf{墨尔本大学} \hfill 澳大利亚墨尔本\\
\textbf{信息技术硕士 (with Distinction)} \hfill 2019年1月 - 2020年12月\\
专业:人工智能,导师:Rui Zhang教授,绩点:88.1 (First Class Honour,专业前2\%)

\vspace{9pt}

\textbf{悉尼大学} \hfill 澳大利亚悉尼\\
\textbf{理学学士} \hfill 2016年1月 - 2018年12月\\
双专业:计算机科学、金融数学与统计学,导师:Simon Poon教授,绩点:86.5 (High Distinction,专业前2\%)

\vspace{9pt}

\textbf{北京理工大学} \hfill 中国北京\\
专业:软件工程,于2016年转学至悉尼大学\hfill 2015年9月 - 2016年1月\\

\vspace{1pt}



%===================

\section{研究主题}

本人的研究主题是将领域自适应算法(Domain Adaptation)用于解决物联网入侵检测问题。所涉及的DA算法涵盖异构DA、多源DA、半监督DA、无监督DA、开集合DA等多种场景。

\vspace{1pt}



%===================

\section{学术发表}

本人发表CCF-A类论文3篇,CCF-B类/清华B类论文4篇,IEEE/ACM Trans在投2篇。发明专利授权7项,在审12项。

%\RNum{1}: IoT Intrusion Detection via Domain Adaptation and its Efficiency Improvement
\begin{enumerate}
  \item Adaptive Bi-recommendation and Self-improving Network for Heterogeneous Domain Adaptation-assisted IoT Intrusion Detection\\
  \textbf{Jiashu Wu}, Yang Wang\textsuperscript{\Letter}, Hao Dai, Chengzhong Xu, Kenneth B. Kent\\
  \textit{IEEE Internet of Things Journal} (\textbf{IEEE IoTJ},清华B类,JCR一区), 2023

  \item Heterogeneous Domain Adaptation for IoT Intrusion Detection: A Geometric Graph Alignment Approach\\
  \textbf{Jiashu Wu}, Hao Dai, Yang Wang\textsuperscript{\Letter}, Kejiang Ye, Chengzhong Xu\\
  \textit{IEEE Internet of Things Journal} (\textbf{IEEE IoTJ},清华B类,JCR一区), 2023

  \item Cost-Efficient Sharing Algorithms for DNN Model Serving in Mobile Edge Networks\\
  Hao Dai, \textbf{Jiashu Wu}, Yang Wang\textsuperscript{\Letter}, Jerome Yen, Yong Zhang, Chengzhong Xu\\
  \textit{IEEE Transactions on Services Computing} (\textbf{IEEE TSC},CCF-A类,JCR一区), 2023

  \item Joint Semantic Transfer Network for IoT Intrusion Detection\\
  \textbf{Jiashu Wu}, Yang Wang\textsuperscript{\Letter}, Binhui Xie, Shuang Li, Hao Dai, Kejiang Ye, Chengzhong Xu\\
  \textit{IEEE Internet of Things Journal} (\textbf{IEEE IoTJ},清华B类,JCR一区), 2022

  \item PackCache: An Online Cost-driven Data Caching Algorithm in the Cloud\\
  \textbf{Jiashu Wu}, Hao Dai, Yang Wang\textsuperscript{\Letter}, Yong Zhang, Dong Huang, Chengzhong Xu\\
  \textit{IEEE Transactions on Computers} (\textbf{IEEE TC},CCF-A类,JCR二区), 2022

  \item Towards Scalable and Efficient Deep-RL in Edge Computing : A Game-based Partition Approach\\
  Hao Dai, \textbf{Jiashu Wu}, Yang Wang\textsuperscript{\Letter}, Chengzhong Xu\\
  \textit{Journal of Parallel and Distributed Computing} (\textbf{JPDC},CCF-B类,JCR一区), 2022

  \item Simultaneous Semantic Alignment Network for Heterogeneous Domain Adaptation\\
  Shuang Li, Binhui Xie, \textbf{Jiashu Wu}, Ying Zhao, Chi Harold Liu\textsuperscript{\Letter}, Zhengming Ding\\
  \textit{ACM International Conference on Multimedia} (\textbf{ACM MM},CCF-A类), 2020, Seattle, Washington, USA

  \item Open Set Dandelion Network for IoT Intrusion Detection\\
  \textbf{Jiashu Wu}, Hao Dai, Yang Wang\textsuperscript{\Letter}, Kenneth B. Kent, Chengzhong Xu\\
  在审,\textit{ACM Transactions on Internet Technology} (\textbf{ACM TOIT},CCF-B类,JCR一区), 2023

  \item HI-CPT: Towards Heterogeneous Verification for Domain Adaptation-based IoT Intrusion Detection\\
  \textbf{Jiashu Wu}, Hao Dai, Yang Wang\textsuperscript{\Letter}, Kejiang Ye, Chengzhong Xu\\
  在审,\textit{IEEE Transactions on Cybernetics} (\textbf{IEEE TCYB},CCF-B类,JCR一区), 2023
\end{enumerate}

\vspace{1pt}



%===================

\section{项目经历}

\textbf{面向物联网入侵检测的领域自适应方法研究} \hfill 博士课题、2019年11月-2020年6月北理工实习

\vspace{2pt}

本课题将领域自适应算法引入物联网入侵检测问题,针对双源域、半监督、无监督、开集合4种数据稀疏场景提出对应的算法。算法设计从自监督学习、概率语义信息、几何空间等多个角度入手,着重克服数据稀疏性、领域异构性,欠适配与负迁移等挑战。课题从入侵检测准确性、迁移机制有效性、算法效率等多角度对算法进行全面评估。所提算法将入侵检测准确率较现有方法提升了4\%-17\%。课题的关键技能包括Python编程、数据特征工程、深度学习算法设计、算法验证以及学术论文撰写。

\vspace{9pt}

\textbf{代价最小化在线云文件缓存算法} \hfill 科技部重点研发专项

\vspace{2pt}

本课题设计了一种代价最小化的在线云文件缓存算法PackCache。算法以在线的形式根据挖掘到的文件访问模式,以单独或打包的形式进行文件传输以满足在线服务请求,并动态进行缓存文件的维护与删除。算法将文件缓存代价降低5\%,且课题从理论层面证明了算法的代价竞争比与竞争比下界吻合。课题的关键技能包括Python编程、算法设计、理论分析以及学术论文撰写。

\vspace{9pt}

\textbf{多索引HDFS遥感大数据存储系统} \hfill 科技部重点研发专项

\vspace{2pt}

本课题设计了基于HDFS的多索引遥感大数据存储与查询系统。系统基于具有冗余数据存储的HDFS,构建了三种地理索引算法并行化的多索引数据重组机制,将数据查询索引时间降低60\%。课题的关键技能包括数据存储算法设计、索引算法设计以及学术论文撰写。

\vspace{1pt}



%===================

\section{获奖情况}
\begin{itemize}
  \item 2022学年度中国科学院大学三好学生
  \item 2019年墨尔本大学Dean's Honours List
  \item 2017\&2018年悉尼大学Dean's List of Excellence in Academic Performance
\end{itemize}

\vspace{1pt}



%===================

\section{技能及语言能力}

编程语言:掌握Python、Java、SQL

语言能力:英语雅思学术类7.0分,大学四级考试665分(阅读部分满分),拥有澳洲五年学习生活经历,可熟练进行英文写作与交流。

\end{document}
