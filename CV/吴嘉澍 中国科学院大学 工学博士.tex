%*****************************%
%      Author: Jiashu Wu      %
% Website: jiashuwu.github.io %
%*****************************%

% For highlighted version, compile on OVERLEAF
% For plain version, compile on MAC

\documentclass[UTF8,letterpaper,11pt]{article}
% Use this for photo version
%\documentclass[UTF8,letterpaper,10.9pt]{article}

\usepackage[UTF8]{ctex}
\usepackage{latexsym}
\usepackage[empty]{fullpage}
\usepackage{titlesec}
\usepackage[usenames,dvipsnames]{color}
\usepackage[hidelinks]{hyperref}
\usepackage{fancyhdr}
\usepackage{marvosym}
\usepackage{geometry}
\usepackage{setspace}
\usepackage{multicol}
\usepackage{graphicx}
\usepackage{multirow}
\geometry{margin=39pt}
\singlespacing
\setCJKmainfont{STSongti-SC-Regular}%MAC, REMOVE ON OVERLEAF

\raggedright
\setlength{\tabcolsep}{0in}

\titleformat{\section}{
  \vspace{-4pt}\scshape\raggedright\large
}{}{0em}{}[\color{black}\titlerule \vspace{-5pt}]

\begin{document}
%\begin{CJK}%OVERLEAF, REMOVE ON MAC

% ********* MAIN CONTENT ********* %

% Non photo version
\iftrue
\begin{tabular*}{\textwidth}{l@{\extracolsep{\fill}}r}
  \huge \textbf{吴嘉澍} & 手机/微信:\href{tel:17801323125}{17801323125}\vspace{2pt}\\
  \Large 中国科学院大学工学博士 & 邮箱:\href{mailto:wujiashu21@mails.ucas.ac.cn}{wujiashu21@mails.ucas.ac.cn}\vspace{2pt}\\
  性别:男\hspace{5mm}生日:1997年6月13日\hspace{5mm}籍贯/生源地:北京市海淀区 & 个人主页:\href{https://jiashuwu.github.io}{jiashuwu.github.io}\\
\end{tabular*}
\fi

% Photo version
\iffalse
\begin{tabular*}{\textwidth}{l@{\extracolsep{\fill}}r}
  & \multirow{4}{*}{\includegraphics[width=2cm,keepaspectratio]{WUJIASHU.jpg}}\\
  \huge \textbf{吴嘉澍}\vspace{2pt}\\
  \Large 中国科学院大学工学博士\vspace{2pt}\\
  性别:男\hspace{5mm}生日:1997年6月13日\hspace{5mm}籍贯/生源地:北京市海淀区\\
  手机/微信:\href{tel:17801323125}{17801323125},邮箱:\href{mailto:wujiashu21@mails.ucas.ac.cn}{wujiashu21@mails.ucas.ac.cn},个人主页:\href{https://jiashuwu.github.io}{jiashuwu.github.io}
\end{tabular*}
\fi

\vspace{1pt}



%===================

\section{\textbf{教育背景}}
\textbf{中国科学院大学} \hfill 中国北京/深圳\\
\textbf{工学博士} \hfill 2021年9月 - 2024年6月\\
专业:计算机应用技术,论文题目:面向物联网入侵检测的领域自适应方法研究\\
研究所:中国科学院深圳先进技术研究院,导师:王洋教授,均分:91.7,\textbf{GPA:3.98}

\vspace{10pt}

\textbf{墨尔本大学} \hfill 澳大利亚墨尔本\\
\textbf{信息技术硕士 (with Distinction)} \hfill 2019年3月 - 2020年12月\\
专业:人工智能,导师:Rui Zhang教授,均分:88.1,\textbf{GPA:4.0} (First Class Honour,前2\%)

\vspace{10pt}

\textbf{悉尼大学} \hfill 澳大利亚悉尼\\
\textbf{理学学士} \hfill 2016年2月 - 2018年12月\\
双专业:计算机科学、金融数学与统计学,导师:Simon Poon教授,均分:86.5,\textbf{GPA:3.96} (High Distinction,前2\%)

\vspace{10pt}

\textbf{北京理工大学} \hfill 中国北京\\
专业:软件工程,于2016年转学至悉尼大学(学信网学籍可查)\hfill 2015年9月 - 2016年1月\\

\vspace{1pt}



%===================

\section{\textbf{博士研究课题}}

将领域自适应算法创新地用于物联网入侵检测。针对无监督、开集合等知识稀缺场景设计高可靠,延迟低的安全监测算法,动态分析千万量级大数据,克服领域异构性,避免欠适配与负迁移效应,保障物联网设备安全。

\vspace{2mm}

\textbf{主要能力}:\textbf{网络大数据分析、模式识别、入侵检测、特征工程、迁移学习、统计分析、算法性能优化、Python编程、学术写作。}

\vspace{1pt}



%===================

\section{\textbf{项目与实习经历}}

\textbf{项目与实习涉及主题}:\textbf{网络大数据分析、迁移学习、大数据缓存优化、数据流分析、数据高效存储等。}

\vspace{2mm}

\textbf{应用领域}:\textbf{网络安全数据分析、大数据分析与存储、物联网、多模态数据分析、模式识别等。}以下为代表项目。

\vspace{2mm}

\textbf{面向物联网入侵检测的领域自适应方法研究} \hfill 博士课题、2019年11月-2020年2月假期北理工实习

\begin{itemize}
  \setlength\itemsep{2.0pt}
  \item \textbf{简介}:创新地将领域自适应算法用于物联网入侵检测,针对知识稀疏场景提出5种\textbf{高可靠、低延迟}的算法。算法从自监督学习、概率语义等角度入手,动态分析\textbf{千万量级大数据},着重克服数据稀疏性、领域异构性,欠适配与负迁移等挑战。\textbf{应用领域}:网络安全数据分析、多模态数据分析与大数据模式识别等。
  \item \textbf{贡献}:所提算法将入侵分析准确率较现有方法大幅提升4\%-17\%,处理大数据具有延迟低的优势,可有效用于IoT安全监测及相关大数据分析识别。
  \item \textbf{技能与成果}:\textbf{网络流量数据分析、入侵检测、深度学习算法设计、特征工程、Python编程、算法性能评估、学术写作。发表CCF-A类/JCR一区论文5篇,专利4项。}
\end{itemize}

\vspace{1pt}

\textbf{代价最小化在线云数据缓存算法研究} \hfill 中科院承接国家科学技术部重点研发专项

\begin{itemize}
  \setlength\itemsep{2.0pt}
  \item \textbf{简介}:设计实现代价最小化在线云数据缓存算法,解决\textbf{在线大数据缓存代价优化}困难的挑战。算法构建了打包式Anticipatory缓存模型。\textbf{应用领域}:为大数据场景提供代价优化、可拓展的分布式数据缓存方案。
  \item \textbf{贡献}:定量分析层面,算法将文件缓存代价降低5\%-11\%,理论分析层面,证明了在线算法与离线算法的代价竞争比与竞争比下界吻合,具有理论价值。
  \item \textbf{技能与成果}:\textbf{大数据缓存优化、数据挖掘与分析、Python编程,发表CCF-A类论文2篇,专利3项。}
\end{itemize}

\vspace{1pt}

\textbf{多索引遥感大数据存储与分析系统} \hfill 科技部重点研发专项、2020年11月-2021年8月中科院深圳先进院实习

\begin{itemize}
  \setlength\itemsep{2.0pt}
  \item \textbf{简介}:设计基于HDFS的多索引\textbf{遥感大数据存储与分析}系统。系统基于分布式存储结构,以并行化多地理索引算法为遥感大数据重组机制,克服遥感数据重组、分析高时延的挑战。\textbf{应用领域}:大数据分析与存储。
  \item \textbf{贡献}:系统将数据分析时间降低60\%,具备数据冗余防丢失、场景适用性强、可拓展、资源开销小等优势。
  \item \textbf{技能与成果}:\textbf{大数据存储系统设计、索引设计、数据库SQL、学术论文撰写,发表JCR一区论文1篇。}
\end{itemize}



%===================

\section{\textbf{学术发表}}

\textbf{已录用CCF-A类/JCR一区论文9篇},在投IEEE/ACM Trans/CCF-A类论文3篇。\textbf{发明专利授权10项},在审9项。\textbf{论文专利主题}:网络数据分析、迁移学习、IoT安全监测、数据缓存与存储等。以下为部分代表论文。

\begin{enumerate}
  \setlength\itemsep{1pt}
  \item Adaptive Bi-recommendation and Self-improving Network for Heterogeneous Domain Adaptation-assisted IoT Intrusion Detection\\
  第一作者,\textit{IEEE Internet of Things Journal} (\textbf{IEEE IoTJ},\textbf{JCR一区},\textbf{IF=10.6}),2023

  \item Heterogeneous Domain Adaptation for IoT Intrusion Detection: A Geometric Graph Alignment Approach\\
  第一作者,\textit{IEEE Internet of Things Journal} (\textbf{IEEE IoTJ},\textbf{JCR一区},\textbf{IF=10.6}),2023

  \item Cost-Efficient Sharing Algorithms for DNN Model Serving in Mobile Edge Networks\\
  第二作者,\textit{IEEE Transactions on Services Computing} (\textbf{IEEE TSC},\textbf{CCF-A类},\textbf{IF=11.0}),2023

  \item Joint Semantic Transfer Network for IoT Intrusion Detection\\
  第一作者,\textit{IEEE Internet of Things Journal} (\textbf{IEEE IoTJ},\textbf{JCR一区},\textbf{IF=10.6}),2022

  \item PackCache: An Online Cost-driven Data Caching Algorithm in the Cloud\\
  第一作者,\textit{IEEE Transactions on Computers} (\textbf{IEEE TC},\textbf{CCF-A类},\textbf{IF=3.7}),2022

  \item Simultaneous Semantic Alignment Network for Heterogeneous Domain Adaptation\\
  导师外第二作者,\textit{ACM International Conference on Multimedia} (\textbf{ACM MM},\textbf{CCF-A类}),2020

  \item MIX-RS: A Multi-indexing System based on HDFS for Remote Sensing Data Storage\\
  第一作者,\textit{Tsinghua Science and Technology} (\textbf{TST},\textbf{JCR一区},\textbf{IF=6.6}),2022

  \item Towards Scalable and Efficient Deep-RL in Edge Computing : A Game-based Partition Approach\\
  第二作者,\textit{Journal of Parallel and Distributed Computing} (\textbf{JPDC},\textbf{JCR一区},\textbf{IF=3.8}),2022

  \item Open Set Dandelion Network for IoT Intrusion Detection\\
  第一作者,在审,\textit{ACM Transactions on Internet Technology},(\textbf{ACM TIOT},\textbf{JCR一区},\textbf{IF=5.3}),2023
\end{enumerate}



%===================

\section{\textbf{获奖情况}}
\vspace{-5mm}
\begin{multicols}{2}
\begin{itemize}
  \setlength\itemsep{1.2pt}
  \item 2023年\textbf{中国科学院院长奖学金}优秀奖\\(\textbf{中科院研究生最高奖,前0.5\%})
  \item 2023年中国科学院大学三好学生标兵
  \item 2022年中国科学院大学三好学生
  \item 2019年墨尔本大学Dean's Honours List
  \item 2017 \& 2018连续两年悉尼大学Dean's List of Excellence in Academic Performance
\end{itemize}
\end{multicols}



%===================

\section{\textbf{技能及语言能力}}

\begin{itemize}
  \setlength\itemsep{2.0pt}
  \item \textbf{技术能力}:掌握Python(机器学习工具PyTorch、Sklearn、数据可视化工具Matplotlib等)、Java、MySQL(数据库优化),可使用Linux操作系统,了解Hadoop的HDFS存储架构。
  \item \textbf{专业基础}:掌握网络、数据分析(数据清洗、特征选择等)、统计、数据库、机器学习、深度学习、操作系统等知识。
  \item \textbf{沟通与写作能力}:雅思学术类(IELTS Academic)7分,大学英语四级CET4 665分(阅读满分),拥有澳洲5年学习生活经历,适应全英文交流写作环境,团队意识、沟通表达能力强。逻辑清晰,可熟练撰写论文、专利与技术报告。
  \item \textbf{工程与科研项目参与}:参与国家科技部及省市级重点研发项目7项,熟悉项目管理、报告撰写等流程。
\end{itemize}

% ********* END OF THE MAIN CONTENT ********* %

%\end{CJK}%OVERLEAF, REMOVE ON MAC
\end{document}
